\begin{exercise}{\emph{(3.1 nel testo)}}
  Per verificare quanto tempo impiega il poliestere a deteriorarsi in
  una discarica, un ricercatore ha preso 10 strisce di poliestere e le
  ha interrate. Cinque strisce scelte a caso sono state estratte dopo
  due settimane e le restanti dopo 3 mesi. Su ogni striscia \`e stata
  misurata la forza di rottura in libbre. Sono state ottenute le
  statistiche seguenti.
  \begin{table}[h]              % 'h' for 'here' position
    \centering
    \begin{tabular}{|c|c|c|c|}
      \hline
      Gruppo & n & media & sd \\\hline
      2 settimane & 5 & 123.8 & 4.6 \\
      3 mesi & 5 & 116.4 & 16.09 \\ \hline
    \end{tabular}
  \end{table}
  Costruire un intervallo di confidenza al 95\% per la differenza
  delle medie assumendo la normalit\`a e l'omoschedasticit\`a.
  Sottoporre a test l'ipotesi di uguaglianza delle medie.
\end{exercise}
Questo il codice che implementa l'esercizio:
\lstinputlisting{r-sources/exercises/chapter-three/three-one.R} Nel
campionamento ripetuto l'intervallo di confidenza riportato sotto
conterr\`a il vero valore del parametro (differenza delle medie) con una
probabilit\`a di copertura del 95\%.
\begin{lstlisting}
  > threeOne()
  $tOss
  [1] 0.9887817

  $meansDifference
  [1] 7.4

  $confidenceInterval
  [1] -9.858037 24.658037

  $pValue
  [1] 0.3517289
\end{lstlisting}
La differenza delle medie non \`e significativa pertanto si mantiene
l'ipotesi che le due medie siano uguali e che il trattamento non ha
avuto effetto.

\begin{exercise}{\emph{(3.3) nel testo}}
  I dati seguenti rappresentano i pesi di 20 uomini partecipanti a un
  programma di perdita di peso, prima e dopo il programma.  Vogliamo
  sapere se in media il programma ha avuto effetto. Fare un test
  oppor- tuno esplicitando le assunzioni.
\end{exercise}
Non possiamo procedere allo studio della differenza delle medie
campionarie dei due gruppi: i due campioni $Y=(Y_1, \ldots, Y_{20})$ e
$Y' = (Y'_1, \ldots, Y'_{20})$ non sono indipendenti, in quanto $(Y_i,
Y'_i)$ sono il peso della persona $i$-esima e quindi il modello non
\`e applicabile direttamente.

Per questo motivo costruiamo il campione $(Y_1-Y'_1, \ldots,
Y_{20}-Y'_{20})$: questo nuovo oggetto viene costruito a partire dai
campioni originari, tra loro indipendenti ma non identicamente
distribuiti, piu' formalmente $(Y_1, \ldots, Y_{20})$ si suppone
distribuito come $N(\mu_1, \sigma^2)$, mentre $(Y'_1, \ldots,
Y'_{20})$ si suppone distribuito come $N(\mu_2, \sigma^2)$.

Per tanto il campione differenza ha distribuzione campionaria data da
$N(\bar{Y} - \bar{Y'}, var(Y) + var(Y') -2cov(Y, Y'))$, chiariamo la
varianza:
\begin{displaymath}
  \begin{split}
    var(Y - Y') &= var(Y + (-Y')) = var(Y) + (-1)^2var(Y') +
    2cov(Y,-Y') \\
    &= var(Y) + var(Y') + 2E\left((Y-E(Y))(-Y'-E(-Y')) \right) = \\
    &= var(Y) + var(Y') + 2E\left((Y-E(Y))(-Y'+E(Y')) \right) = \\
    &= var(Y) + var(Y') - 2E\left((Y-E(Y))(Y'-E(Y')) \right) = \\
    &= var(Y) + var(Y') - 2cov(Y,Y')
  \end{split}
\end{displaymath}
Questo il codice che implementa l'esercizio:
\lstinputlisting{r-sources/exercises/chapter-three/three-three.R}
Abbiamo fissato l'ipotesi nulla $H_0$ che la media del nuovo campione
(ovvero delle differenze tra prima e dopo la dieta) sia uguale a 0 ad
un livello del 95\%, con una ipotesi alternativa unilaterale (in
quanto speriamo di avere una perdita di peso):
\begin{lstlisting}
> threeThree()
$sample
[1]  3.8 -5.5  8.9  0.0  8.0  5.2  7.7  7.5  5.0  0.8  0.0  2.6  9.0 -3.6  0.1
[16]  0.5  2.0  0.0  0.9  0.0

$automaticTest

One Sample t-test

data:  sample 
t = 2.8734, df = 19, p-value = 0.004865
alternative hypothesis: true mean is greater than 0 
95 percent confidence interval:
1.053314      Inf 
sample estimates:
mean of x 
2.645 


$empiricalMean
[1] 2.645

$tOss
[1] 2.873404

$confidenceInterval
[1] 0.718348 4.571652

$pValue
[1] 0.004865231
\end{lstlisting}
La differenza \`e altamente significativa pertanto si rifiuta $H_0$ e
il trattamento ha avuto effetto.

\begin{exercise}
  In un ospedale 485 pazienti sono stati sottoposti a due tipi di
  chirurgia, ginecologica (Gruppo 1) e addominale (Gruppo 2), ed \`e
  stata registrata la presenza successiva di complicazioni
  postoperatorie. Costruire un intervallo di confidenza per la
  differenza delle proporzioni. Sottoporre a test che non ci siano
  differenze tra i due gruppi.
\end{exercise}
Si implementa direttamente:
\begin{lstlisting}
  > prop.test(c(235, 210), c(240, 245), correct = FALSE, conf.level = 0.95, alternative = "two.sided")

	2-sample test for equality of proportions without continuity
	correction

data:  c(235, 210) out of c(240, 245) 
X-squared = 23.8555, df = 1, p-value = 1.038e-06
alternative hypothesis: two.sided 
95 percent confidence interval:
 0.07462716 0.16942046 
sample estimates:
   prop 1    prop 2 
0.9791667 0.8571429 

> zOss <- sqrt(23.8555)
> zOss
[1] 4.884209
\end{lstlisting}
L'intervallo di confidenza al livello 95\% per la differenza di
proporzioni \`e $(0.075, 0.17)$: quest'intervallo delimita
l'incremento nella probabilit\`a di non avere complicazioni
postoperatorie per pazienti sottoposti alla chirurgia
addominale. Inoltre la differenza \`e altamente significativa, si
rifiuta l'ipotesi nulla $H_0$ che non vi sia differenza tra le due
proporzioni, quindi la domanda iniziale ``la chirurgia ginecologica ha
meno complicazioni postoperatorie'' ha risposta affermativa.
