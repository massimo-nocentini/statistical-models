\begin{exercise}{\emph{(3.1 nel testo)}}
  Per verificare quanto tempo impiega il poliestere a deteriorarsi in
  una discarica, un ricercatore ha preso 10 strisce di poliestere e le
  ha interrate. Cinque strisce scelte a caso sono state estratte dopo
  due settimane e le restanti dopo 3 mesi. Su ogni striscia \`e stata
  misurata la forza di rottura in libbre. Sono state ottenute le
  statistiche seguenti.
  \begin{table}[h]              % 'h' for 'here' position
    \centering
    \begin{tabular}{|c|c|c|c|}
      \hline
      Gruppo & n & media & sd \\\hline
      2 settimane & 5 & 123.8 & 4.6 \\
      3 mesi & 5 & 116.4 & 16.09 \\ \hline
    \end{tabular}
  \end{table}
  Costruire un intervallo di confidenza al 95\% per la differenza
  delle medie assumendo la normalit\`a e l'omoschedasticit\`a.
  Sottoporre a test l'ipotesi di uguaglianza delle medie.
\end{exercise}
Questo il codice che implementa l'esercizio:
\lstinputlisting{r-sources/exercises/chapter-three/three-one.R} Nel
campionamento ripetuto l'intervallo di confidenza riportato sotto
conterr\`a il vero valore del parametro (differenza delle medie) con una
probabilit\`a di copertura del 95\%.
\begin{lstlisting}
  > threeOne()
  $tOss
  [1] 0.9887817

  $meansDifference
  [1] 7.4

  $confidenceInterval
  [1] -9.858037 24.658037

  $pValue
  [1] 0.3517289
\end{lstlisting}
La differenza delle medie non \`e significativa pertanto si mantiene
l'ipotesi che le due medie siano uguali e che il trattamento non ha
avuto effetto.

\begin{exercise}{\emph{(3.3) nel testo}}
  I dati seguenti rappresentano i pesi di 20 uomini partecipanti a un
  programma di perdita di peso, prima e dopo il programma.  Vogliamo
  sapere se in media il programma ha avuto effetto. Fare un test
  oppor- tuno esplicitando le assunzioni.
\end{exercise}
Non possiamo procedere allo studio della differenza delle medie
campionarie dei due gruppi: i due campioni $Y=(Y_1, \ldots, Y_{20})$ e
$Y' = (Y'_1, \ldots, Y'_{20})$ non sono indipendenti, in quanto $(Y_i,
Y'_i)$ sono il peso della persona $i$-esima e quindi il modello non
\`e applicabile direttamente.

Per questo motivo costruiamo il campione $(Y_1-Y'_1, \ldots,
Y_{20}-Y'_{20})$: questo nuovo oggetto viene costruito a partire dai
campioni originari, tra loro indipendenti ma non identicamente
distribuiti, piu' formalmente $(Y_1, \ldots, Y_{20})$ si suppone
distribuito come $N(\mu_1, \sigma^2)$, mentre $(Y'_1, \ldots,
Y'_{20})$ si suppone distribuito come $N(\mu_2, \sigma^2)$.

Per tanto il campione differenza ha distribuzione campionaria data da
$N(\bar{Y} - \bar{Y'}, var(Y) + var(Y') -2cov(Y, Y'))$, chiariamo la
varianza:
\begin{displaymath}
  \begin{split}
    var(Y - Y') &= var(Y + (-Y')) = var(Y) + (-1)^2var(Y') +
    2cov(Y,-Y') \\
    &= var(Y) + var(Y') + 2E\left((Y-E(Y))(-Y'-E(-Y')) \right) = \\
    &= var(Y) + var(Y') + 2E\left((Y-E(Y))(-Y'+E(Y')) \right) = \\
    &= var(Y) + var(Y') - 2E\left((Y-E(Y))(Y'-E(Y')) \right) = \\
    &= var(Y) + var(Y') - 2cov(Y,Y')
  \end{split}
\end{displaymath}
Questo il codice che implementa l'esercizio:
\lstinputlisting{r-sources/exercises/chapter-three/three-three.R}
Abbiamo fissato l'ipotesi nulla $H_0$ che la media del nuovo campione
(ovvero delle differenze tra prima e dopo la dieta) sia uguale a 0 ad
un livello del 95\%, con una ipotesi alternativa unilaterale (in
quanto speriamo di avere una perdita di peso):
\begin{lstlisting}
> threeThree()
$sample
[1]  3.8 -5.5  8.9  0.0  8.0  5.2  7.7  7.5  5.0  0.8  0.0  2.6  9.0 -3.6  0.1
[16]  0.5  2.0  0.0  0.9  0.0

$automaticTest

One Sample t-test

data:  sample 
t = 2.8734, df = 19, p-value = 0.004865
alternative hypothesis: true mean is greater than 0 
95 percent confidence interval:
1.053314      Inf 
sample estimates:
mean of x 
2.645 


$empiricalMean
[1] 2.645

$tOss
[1] 2.873404

$confidenceInterval
[1] 0.718348 4.571652

$pValue
[1] 0.004865231
\end{lstlisting}
La differenza \`e altamente significativa pertanto si rifiuta $H_0$ e
il trattamento ha avuto effetto.

\begin{exercise}{\emph{(3.4 nel testo)}}
  In un ospedale 485 pazienti sono stati sottoposti a due tipi di
  chirurgia, ginecologica (Gruppo 1) e addominale (Gruppo 2), ed \`e
  stata registrata la presenza successiva di complicazioni
  postoperatorie.
  \begin{table}[h]              % 'h' for 'here' position
    \centering
    \begin{tabular}{|c|c|c|c|}
      \hline
      operaz./compl. & no & si & totale \\\hline
      ginecologica & 235 & 5 & 240 \\
      addominale & 210 & 35 & 245 \\ \hline
    \end{tabular}
  \end{table}
  Costruire un intervallo di confidenza per la
  differenza delle proporzioni. Sottoporre a test che non ci siano
  differenze tra i due gruppi.
\end{exercise}
Si implementa direttamente:
\begin{lstlisting}
  > prop.test(c(235, 210), c(240, 245), correct = FALSE,
  conf.level = 0.95, alternative = "two.sided")

  2-sample test for equality of proportions without
  continuity correction

data:  c(235, 210) out of c(240, 245) 
X-squared = 23.8555, df = 1, p-value = 1.038e-06
alternative hypothesis: two.sided 
95 percent confidence interval:
 0.07462716 0.16942046 
sample estimates:
   prop 1    prop 2 
0.9791667 0.8571429 

> zOss <- sqrt(23.8555)
> zOss
[1] 4.884209
\end{lstlisting}
L'intervallo di confidenza al livello 95\% per la differenza di
proporzioni \`e $(0.075, 0.17)$: quest'intervallo delimita
l'incremento nella probabilit\`a di non avere complicazioni
postoperatorie per pazienti sottoposti alla chirurgia
ginecologica. Inoltre la differenza \`e altamente significativa, si
rifiuta l'ipotesi nulla $H_0$ che non vi sia differenza tra le due
proporzioni, quindi la domanda iniziale ``la chirurgia ginecologica ha
meno complicazioni postoperatorie'' ha risposta affermativa.

\begin{exercise}{\emph{(3.5 nel testo)}}
  La mancata risposta nelle indagini campionarie puo' variare in base
  al periodo dell'anno. L'ISTAT ha analizzato il numero di chiamate su
  un numero telefonico attivo che non hanno avuto risposta in due
  inchieste telefoniche, la prima svolta da gennaio ad aprile e la
  seconda svolta dal 1 luglio al 31 agosto.
  \begin{table}[h]              % 'h' for 'here' position
    \centering
    \begin{tabular}{|c|c|c|}
      \hline
      periodo & chiamate totali & Senza risposta \\\hline
      1/1 - 14/4 & 1558 & 333 \\
      1/7 - 31/8 & 2075 & 861  \\ \hline
    \end{tabular}
  \end{table}
  Di quanto \`e superiore la proporzione di chiamate senza risposta
  nel secondo periodo rispetto al primo periodo? Trovare un intervallo
  di confidenza al 99\%.  
\end{exercise}
Prima di applicare uno dei modelli per il confronto dei due gruppi
manipoliamo la matrice iniziale in modo da poter rispondere alla
domanda posta, scambiando le due righe, le due colonne e esplicitando
i successi e gli insuccessi (in quanto vogliamo identificare come
``successi'' le \emph{chiamate senza risposta}), ottenendo:
  \begin{table}[h]              % 'h' for 'here' position
    \centering
    \begin{tabular}{|c|c|c|}
      \hline
      periodo & senza risposta & con risposta\\\hline
      1/7 - 31/8 & 861 & 1214 \\
      1/1 - 14/4 & 333 & 1225 \\\hline
    \end{tabular}
  \end{table}
  Adesso possiamo studiare il log del rapporto delle quote in quanto
  invariante a scambio di righe e colonne (quello che abbiamo appena
  fatto sopra). Questa la funzione che implementa l'esercizio:
  \lstinputlisting{r-sources/exercises/chapter-three/three-five-automatic-oddsRation.R}
  Otteniamo i seguenti risultati:
  \begin{lstlisting}
    > threeFiveAutomatic(rbind(c(861, 2075 - 861), c(333, 1558 - 333)))
    $matrix
    [,1] [,2]
    [1,]  861 1214
    [2,]  333 1225

    $confidenceIntervalOddsRatio
    [1] 2.144327 3.174400

    $confidenceIntervalRisk
    [1] 1.595596 2.362074
\end{lstlisting}
L'intervallo di confidenza al 99\% \`e $(2.14, 3.17)$. Per il calcolo
dell'intervallo per il rischio relativo abbiamo usato l'identit\`a:
\begin{displaymath}
  RR = OR\frac{1-p_1}{1-p_2}
\end{displaymath}
Il rishio di chiamate \emph{senza} risposta nel periodo estivo varia
da 1.6 a 2.4 volte rispetto al rischio nel periodo invernale (infatti
\`e coerente che il numero di successi, dove per successo
identifichiamo una chiamata \emph{senza} risposta con questa matrice,
sia diverso). Possiamo concludere che con questa matrice \textbf{\`e
  possibile} apprezzare la relazione che intercorre tra i due gruppi
rispetto alle chiamate \emph{senza} risposta.

\`E interessante fare un confronto con la matrice originale:
\begin{lstlisting}
  > threeFiveAutomatic(rbind(c(1558 - 333, 333), c(2075 - 861, 861)))
  $matrix
  [,1] [,2]
  [1,] 1225  333
  [2,] 1214  861

  $confidenceIntervalOddsRatio
  [1] 2.144327 3.174400

  $confidenceIntervalRisk
  [1] 1.104543 1.635134 
\end{lstlisting}
Osserviamo che l'intervallo per gli \emph{odd ratio} \`e uguale
(infatti gli odd ratio sono invarianti per rivoluzioni di matrice,
come abbiamo ricordato prima), mentre l'intervallo per i rischi
cambia: adesso dobbiamo interpretare che il rischio di avere chiamate
\emph{con} risposta nel periodo invernale varia da 1.1 a 1.6 volte
rispetto al rischio nel periodo estivo (infatti \`e coerente che il
numero di successi, dove per successo identifichiamo una chiamata
\emph{con} risposta con questa matrice, sia approssimativamente
uguale). Possiamo concludere che con questa matrice \textbf{non \`e
  possibile} apprezzare la relazione che intercorre tra i due gruppi
rispetto alle chiamate \emph{con} risposta.
