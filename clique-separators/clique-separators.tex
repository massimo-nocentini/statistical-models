% $Header$

\documentclass{beamer}

% This file is a solution template for:

% - Giving a talk on some subject.
% - The talk is between 15min and 45min long.
% - Style is ornate.



% Copyright 2004 by Till Tantau <tantau@users.sourceforge.net>.
%
% In principle, this file can be redistributed and/or modified under
% the terms of the GNU Public License, version 2.
%
% However, this file is supposed to be a template to be modified
% for your own needs. For this reason, if you use this file as a
% template and not specifically distribute it as part of a another
% package/program, I grant the extra permission to freely copy and
% modify this file as you see fit and even to delete this copyright
% notice. 


\mode<presentation>
{
  \usetheme{Pittsburgh}
  \usecolortheme{dove}
  \usefonttheme{professionalfonts}
  \setbeamertemplate{blocks}[rounded][shadow=false]
  \setbeamercolor{block title}{fg=structure,bg=white}

}


\usepackage[english]{babel}
\usepackage[utf8]{inputenc}
\usepackage[T1]{fontenc}
\usepackage{CJKutf8}
\usepackage{amsfonts}
\usepackage{color}
\usepackage{pstricks,pst-text}
\usepackage{pst-node,pst-tree}
\usepackage{concrete}

\usepackage{amsmath}%
\usepackage{amsthm}
\usepackage{amssymb}%
\usepackage{graphicx}
\usepackage{float}
\usepackage{caption}
\usepackage{bbold}
% Or whatever. Note that the encoding and the font should match. If T1
% does not look nice, try deleting the line with the fontenc.


\title[Algebraic gfs for languages avoiding Riordan patterns] % (optional, use only with long paper titles)
{Algebraic generating functions for languages  avoiding Riordan patterns}

%\subtitle {Presentation Subtitle} % (optional)

\author[Merlini, Nocentini] % (optional, use only with lots of authors)
{Donatella Merlini \and Massimo Nocentini}
% - Use the \inst{?} command only if the authors have different
%   affiliation.

\institute[Universities of Somewhere and Elsewhere] % (optional, but mostly needed)
{
  Dipartimento di Statistica, Informatica, Applicazioni \\
  University of Florence, Italy
}
% - Use the \inst command only if there are several affiliations.
% - Keep it simple, no one is interested in your street address.

\date[Short Occasion] % (optional)
{\today \\ AORC2017}

\subject{
We study the languages  $\mathfrak{L}^{[\mathfrak{p}]}\subset \{0,1\}^*$ of
binary words $w$ avoiding a given pattern $\mathfrak{p}$ such that $|w|_0\leq
|w|_1$ for any $w\in \mathfrak{L}^{[\mathfrak{p}]},$ where  $|w|_0$ and $|w|_1$
correspond to the number of bits $1$ and $0$ in the word $w$, respectively.  In
particular, we concentrate on  patterns $\mathfrak{p}$ related to the concept
of Riordan arrays. These languages are not regular and can be enumerated by
algebraic generating functions corresponding to many integer sequences which
are unknown in the OEIS.  We give explicit formulas for these generating
functions expressed in terms of the autocorrelation polynomial of
$\mathfrak{p}$ and also give explicit formulas for the coefficients of some
particular patterns, algebraically and combinatorially.}
% This is only inserted into the PDF information catalog. Can be left
% out. 



% If you have a file called "university-logo-filename.xxx", where xxx
% is a graphic format that can be processed by latex or pdflatex,
% resp., then you can add a logo as follows:

% \pgfdeclareimage[height=0.5cm]{university-logo}{university-logo-filename}
% \logo{\pgfuseimage{university-logo}}



% Delete this, if you do not want the table of contents to pop up at
% the beginning of each subsection:
\AtBeginSection[]
{
  \begin{frame}<beamer>{Outline}
    \tableofcontents[currentsection]
  \end{frame}
}


% If you wish to uncover everything in a step-wise fashion, uncomment
% the following command: 

%\beamerdefaultoverlayspecification{<+->}


\begin{document}

\begin{frame}
  \titlepage
\end{frame}

\begin{frame}{Outline}
  \tableofcontents
  % You might wish to add the option [pausesections]
\end{frame}


% Since this a solution template for a generic talk, very little can
% be said about how it should be structured. However, the talk length
% of between 15min and 45min and the theme suggest that you stick to
% the following rules:  

% - Exactly two or three sections (other than the summary).
% - At *most* three subsections per section.
% - Talk about 30s to 2min per frame. So there should be between about
%   15 and 30 frames, all told.

\section{Introduction}

\begin{frame}
\frametitle{Definition in terms of $d(t)$ and $h(t)$}
\begin{itemize}
\item A \emph{Riordan array} is a pair
$$D=\mathcal{R}(d(t),\ h(t))$$
in which $d(t)$ and $h(t)$ are formal power series such that $d(0)\neq 0$ and  $h(0)= 0$
\item if $h^\prime(0)\neq 0$ the Riordan array is called \emph{proper}
\item it denotes an infinite, lower triangular array $(d_{n,k})_{n,k\in N}$ where:
$$d_{n,k}=[t^n]d(t)h(t)^k$$
\end{itemize}
\end{frame}

\begin{frame}\frametitle{The $A$ and $Z$ sequences}
An alternative definition, is in terms of the so-called $A$-sequence and $Z$-sequence, with generating functions $A(t)$ and $Z(t)$ satisfying
the relations:
$$ h(t)=tA(h(t)), \quad d(t)={d_0\over 1-tZ(h(t))} \quad \mbox{with} \quad d_0=d(0).$$
$$
d_{n+1,k+1}=a_0d_{n,k}+a_1d_{n,k+1}+a_2d_{n,k+2}+\cdots
$$
$$
d_{n+1,0}=z_0d_{n,0}+z_1d_{n,1}+z_2d_{n,2}+\cdots
$$
\end{frame}


\begin{frame}\frametitle{The $A$-matrix {\tiny [MSRV97]}}
$$ d_{n+1,k+1}={\underset{i\geq0}{ \displaystyle\sum }}{\underset{j\geq0}{ \displaystyle\sum }}\alpha_{i,j}d_{n-i,k+j}+ \underset{j\geq
0}{\displaystyle\sum}\rho_jd_{n+1,k+j+2}
$$
Matrix $(\alpha_{i,j})_{i,j\in \mathbb{N}}$  is called the
$A$-matrix of the Riordan array. If,  for $i \geq 0:$
$$P^{[i]}(t)=\alpha_{i,0}+\alpha_{i,1}t+\alpha_{i,2}t^2+\alpha_{i,3}t^3+\ldots$$
 and $Q(t)$ is the generating function for the
sequence $(\rho_j)_{j\in \mathbb{N}}$, then we have:
$$
\dfrac{h(t)}{t}=\underset{i\geq 0}{\displaystyle\sum}t^iP^{[i]}(h(t))+\dfrac{h(t)^{2}}{t}Q(h(t))
$$
$$
A(t) =\underset{i\geq 0}{\displaystyle\sum}t^iA(t)^{-i}P^{[i]}(t)+tA(t)Q(t)
$$
\end{frame}

\section{Binary words avoiding patterns}

\begin{frame} \frametitle{Binary words avoiding a pattern}
\begin{itemize}
\item We consider the language $\mathcal{L}^{[\mathfrak{p}]}$ of binary words
with no occurrence of a pattern  $\mathfrak{p}=p_0\cdots p_{h-1}$
\item  The problem of determining the generating function  counting
the number of words \emph{with respect to their length} has been studied
by several authors:

\begin{enumerate}
\item L.~J. Guibas and M.~Odlyzko.
 Long repetitive patterns in random sequences.
{\em Zeitschrift f\"{u}r Wahrscheinlichkeitstheorie}, 53:241--262,
  1980.
\item R.~Sedgewick and P.~Flajolet.
 {\em An {I}ntroduction to the {A}nalysis of {A}lgorithms}.
Addison-Wesley, Reading, MA, 1996.
\end{enumerate}
\item The fundamental notion is that of the \emph{autocorrelation vector} of
bits $c=(c_0,\ldots ,c_{h-1})$ associated to a given $\mathfrak{p}$
\end{itemize}
\end{frame}

\begin{frame} \frametitle{The pattern $\mathfrak{p}=10101$}
\begin{center}
\begin{tabular}{ccccc|cccccccc}
  $1$ & $0$ & $1$ & $0$ & $1$ & \multicolumn{6}{l}{Tails} & $c_{i}$  \\
  \hline
  $1$ & $0$ & $1$ & $0$ & $1$ & &   &   &   &   &   &    $1$ \\
    & $1$ & $0$ & $1$ & $0$ & $1$ &  &   &   &   &   &    $0$ \\
    &   & $1$ & $0$ & $1$ & $0$ & $1$ & &   &   &   &    $1$ \\
    &   &   & $1$ & $0$ & $1$ & $0$ & $1$ &  &   &   &    $0$ \\
    &   &   &   & $1$ & $0$ & $1$ & $0$ & $1$ &  &   &    $1$\\
\end{tabular}
\end{center}
The autocorrelation vector is then $c=(1,0,1,0,1)$ and  
$C^{[\mathfrak{p}]}(t)=1+t^{2}+t^{4}$ is the associated
autocorrelation polynomial
\end{frame}

\begin{frame} \frametitle{Count respect bits $1$ and $0$}
The gf counting the number $F_{n}$ of binary words with 
length $n$ not containing the pattern $\mathfrak{p}$ is
\begin{displaymath}
    F(t) = \frac{C^{[\mathfrak{p}]}(t)}{t^{h} + (1-2t)C^{[\mathfrak{p}]}(t)}
\end{displaymath}
Taking into account the number of bits $1$ and $0$ in $\mathfrak{p}$:
\begin{displaymath}
    F^{[\mathfrak{p}]}(x, y) = \frac{C^{[\mathfrak{p}]}(x,y)}{x^{n_{1}^{[\mathfrak{p}]}} y^{n_{0}^{[\mathfrak{p}]}}
        + (1-x-y)C^{[\mathfrak{p}]}(x, y)}
\end{displaymath}
where $h = {n_{0}^{[\mathfrak{p}]}}+{n_{1}^{[\mathfrak{p}]}}$ and $C^{[\mathfrak{p}]}(x, y)$
is the bivariate autocorrelation polynomial. Moreover, 
$F_{n, k}^{[\mathfrak{p}]} =[x^n y^k]F^{[\mathfrak{p}]}(x,y)$ denotes the number of binary 
words avoiding the pattern $\mathfrak{p}$ with $n$ bits $1$ and $k$ bits $0$
\end{frame}

\begin{frame}\frametitle{An example with $\mathfrak{p}= 10101$}
Since $C^{[\mathfrak{p}]}(x,y)=1+xy+x^2y^2$ we have:
$$F^{[\mathfrak{p}]}(x,y)={1+xy+x^2y^2 \over (1-x-y)(1+xy+x^2y^2)+x^3y^2}. $$
$$
\begin{array}{c|cccccccc}
n/k  & 0 & 1 & 2 & 3 & 4 &5 &6 &7  \\ \cline{1-9} 0 & {\bf\color{red} 1} & {\bf\color{green} 1} & {\bf\color{green}1} & {\bf\color{green}1}&
{\bf\color{green}1}
&  {\bf\color{green}1} & {\bf\color{green}1} & {\bf\color{green}1}\\
1 & {\bf\color{blue} 1} & {\bf\color{red}2} & 3 & 4 & 5 & 6 & 7 & 8 \\
2 & {\bf\color{blue} 1} & 3 & {\bf\color{red}6} & 10 & 15 & 21 & 28 & 36\\
3 & {\bf\color{blue} 1} & 4 & 9 & {\bf\color{red}18} & 32 & 52& 79 & 114\\
4 & {\bf\color{blue} 1} & 5 & 13 & 30 & {\bf\color{red}60} & 109& 184 & 293\\
5 & {\bf\color{blue} 1} & 6 & 18 & 46 & 102 & {\bf\color{red}204} & 377 & 654\\
6 & {\bf\color{blue} 1} & 7 & 24 & 67 & 163 & 354& {\bf\color{red}708} & 1324\\
7 & {\bf\color{blue} 1} & 8 & 31 & 94 & 248 &580& 1245 &{\bf\color{red}2490}
\end{array}
$$
\end{frame}


\begin{frame}\frametitle{...the lower and upper triangular parts}
\begin{columns}
\begin{column}{5cm}
$$
\begin{array}{c|cccccc}
n/k  & 0 & 1 & 2 & 3 & 4 &5   \\ \cline{1-7}
0 & {\bf\color{red}1} & & & & & \\
1 & {\bf\color{red}2} & {\bf\color{blue} 1} & & &  & \\
2 & {\bf\color{red}6}& 3 & {\bf\color{blue} 1} & & & \\
3 & {\bf\color{red}18} & 9 & 4 & {\bf\color{blue} 1} & & \\
4 & {\bf\color{red}60} & 30 & 13 & 5 & {\bf\color{blue} 1}& \\
5 & {\bf\color{red}204} & 102 & 46 & 18 & 6 &{\bf\color{blue} 1} \\
\end{array}
$$
$(n,k)\mapsto(n,n-k)$ if $k\leq n$
\end{column}
\begin{column}{5cm}
$$
\begin{array}{c|cccccc}
n/k  & 0 & 1 & 2 & 3 & 4 &5   \\ \cline{1-7}
0 & {\bf\color{red}1} & & & & & \\
1 & {\bf\color{red}2} & {\bf\color{green}1} & & &  & \\
2 & {\bf\color{red}6}& 3 & {\bf\color{green}1} & & & \\
3 & {\bf\color{red}18} & 10 & 4 & {\bf\color{green}1} & & \\
4 & {\bf\color{red}60} & 32 & 15 & 5 & {\bf\color{green}1}&  \\
5 & {\bf\color{red}204} & 109 & 52 & 21 & 6 &{\bf\color{green}1} \\
\end{array}
$$
$(n,k)\mapsto(k,k-n)$ if $n\leq k$
\end{column}
\end{columns}
\end{frame}

\begin{frame}\frametitle{Matrices ${{R}^{[\mathfrak{p}]}}$ and  ${{R}^{[\bar{\mathfrak{p}]}}}$}
\begin{itemize}
\item Let $R_{n,k}^{[\mathfrak{p}]}=F_{n,n-k}^{[\mathfrak{p}]}$ with $k\leq n.$
In other words, $R_{n,k}^{[\mathfrak{p}]}$ counts the number of words avoiding
$\mathfrak{p}$ with $n$ bits $1$ and $n-k$  bits $0$ \item Let
$\bar{\mathfrak{p}}=\bar{p}_{0}\ldots\bar{p}_{h-1}$ be the $\mathfrak{p}$'s conjugate, 
where $\bar{p}_{i} = 1-p_{i}$
\item We obviously have
$R_{n,k}^{[\bar{\mathfrak{p}}]}=F_{n,n-k}^{[\bar{\mathfrak{p}}]}=F_{k,k-n}^{[\mathfrak{p}]}$.
Therefore, the matrices ${{R}^{[\mathfrak{p}]}}$ and
${{R}^{[\bar{\mathfrak{p}]}}}$ represent the lower and upper triangular part of
the array ${{F}^{[\mathfrak{p}]}},$ respectively
\end{itemize}

\end{frame}

\section{Riordan patterns}

\begin{frame}\frametitle{Riordan patterns {\tiny [MS11]}}
\begin{itemize}
\item
When matrices
${{R}^{[\mathfrak{p}]}}$ and ${{R}^{[\bar{\mathfrak{p}]}}}$ are (both) Riordan arrays?
\item
We say that $\mathfrak{p}=p_0...p_{h-1}$ is a Riordan pattern if
and only if
$$C^{[\mathfrak{p}]}(x,y)=C^{[\mathfrak{p}]}(y,x)=
\sum_{i=0}^{\lfloor(h-1)/2\rfloor}c_{2i}x^iy^i$$
provided that $\left|n_1^{[\mathfrak{p}]}-n_0^{[\mathfrak{p}]}\right|\in \left\{0,1\right\}$
\end{itemize}
\begin{enumerate}
\item {\small D. Merlini and R. Sprugnoli. Algebraic aspects of some Riordan arrays related to binary words avoiding a pattern.
{\em Theoretical Computer Science}, 412 (27), 2988-3001, 2011.}
\end{enumerate}

\end{frame}


%\begin{frame}\frametitle{Theorem 1}
%The matrices ${\cal{R}^{[\mathfrak{p}]}}$ and
%${\cal{R}^{[\bar{\mathfrak{p}]}}}$ are both Riordan arrays
%${\cal{R}^{[\mathfrak{p}]}}=(d^{[\mathfrak{p}]}(t),h^{[\mathfrak{p}]}(t))$
% and ${\cal{R}^{[\bar{\mathfrak{p}]}}}=(d^{[\bar{\mathfrak{p}}]}(t),h^{[\bar{\mathfrak{p}}]}(t))$
%if and only if  $\mathfrak{p}$ is a Riordan pattern. Moreover we have:
%\begin{alertblock}{ }
%$$d^{[\mathfrak{p}]}(t)=d^{[\bar{\mathfrak{p}}]}(t)=
%[x^0]F\left(x,\dfrac{t}{x}\right)=\dfrac{1}{2\pi i}\displaystyle\oint{F\left(x,\dfrac{t}{x}\right)\dfrac{dx}{x}}
%$$
%and {\tiny $$h^{[\mathfrak{p}]}(t)={1- \sum_{i =
%0}^{n_1^{\mathfrak{p}}-1} \alpha_{i,1}t^{i+1}-
%\sqrt{(1-\sum_{i=0}^{n_1^{\mathfrak{p}}-1} \alpha_{i,1}t^{i+1})^2
%- 4\sum_{i = 0}^{n_1^{\mathfrak{p}}-1} \alpha_{i,0}t^{i+1}(\sum_{i
%= 0}^{n_1^{\mathfrak{p}}-1} \alpha_{i,2}t^{i+1}+1)} \over
%2(\sum_{i = 0}^{n_1^{\mathfrak{p}}-1} \alpha_{i,2}t^{i+1}+1)} $$}
%\end{alertblock}
%\end{frame}


\begin{frame}\frametitle{Theorem 1}
Matrices 
\begin{displaymath}
{{R}^{[\mathfrak{p}]}}=(d^{[\mathfrak{p}]}(t),h^{[\mathfrak{p}]}(t)), \quad
{{R}^{[\bar{\mathfrak{p}]}}}=(d^{[\bar{\mathfrak{p}}]}(t),h^{[\bar{\mathfrak{p}}]}(t))
\end{displaymath}
are both RAs $\leftrightarrow$  $\mathfrak{p}$ is a Riordan pattern.

\begin{block}{}
By specializing this result  to the cases
$\left|n_1^{[\mathfrak{p}]}-n_0^{[\mathfrak{p}]}\right|\in \{0,1\}$ and by
setting
$C^{[\mathfrak{p}]}(t)=C^{[\mathfrak{p}]}\left(\sqrt{t},\sqrt{t}\right)=\sum_{i
\geq 0}c_{2i}t^i,$ we have
\end{block}

\end{frame}

\begin{frame}\frametitle{Theorem 1: the case $n_1^{[\mathfrak{p}]}-n_0^{[\mathfrak{p}]}=1$}
$$d^{[\mathfrak{p}]}(t)={C^{[\mathfrak{p}]}(t)
\over \sqrt{C^{[\mathfrak{p}]}(t)^2-4tC^{[\mathfrak{p}]}(t)(C^{[\mathfrak{p}]}(t)-t^{n_0^{\mathfrak{p}}})}}, $$

$$h^{[\mathfrak{p}]}(t)={C^{[\mathfrak{p}]}(t) -\sqrt{C^{[\mathfrak{p}]}(t)^2-4tC^{[\mathfrak{p}]}(t)(C^{[\mathfrak{p}]}(t)-t^{n_0^{\mathfrak{p}}})}
\over 2 C^{[\mathfrak{p}]}(t)}.$$
\end{frame}

\begin{frame}\frametitle{Theorem 1: the case $n_1^{[\mathfrak{p}]}-n_0^{[\mathfrak{p}]}=0$}
$$d^{[\mathfrak{p}]}(t)={C^{[\mathfrak{p}]}(t)
\over \sqrt{( C^{[\mathfrak{p}]}(t)+t^{n_0^{\mathfrak{p}}})^2-4tC^{[\mathfrak{p}]}(t)^2}}, $$

$$h^{[\mathfrak{p}]}(t)=
{C^{[\mathfrak{p}]}(t)+ t^{n_0^{\mathfrak{p}}} - \sqrt{( C^{[\mathfrak{p}]}(t)+t^{n_0^{\mathfrak{p}}})^2-4tC^{[\mathfrak{p}]}(t)^2} \over 2
C^{[\mathfrak{p}]}(t)}.$$
\end{frame}

\begin{frame}\frametitle{Theorem 1: the case $n_0^{[\mathfrak{p}]}-n_1^{[\mathfrak{p}]}=1$}
$$d^{[\mathfrak{p}]}(t)={C^{[\mathfrak{p}]}(t)
\over \sqrt{C^{[\mathfrak{p}]}(t)^2-4tC^{[\mathfrak{p}]}(t)(C^{[\mathfrak{p}]}(t)-t^{n_1^{\mathfrak{p}}})}}, $$

$$h^{[\mathfrak{p}]}(t)={C^{[\mathfrak{p}]}(t) -\sqrt{C^{[\mathfrak{p}]}(t)^2-4tC^{[\mathfrak{p}]}(t)(C^{[\mathfrak{p}]}(t)-t^{n_1^{\mathfrak{p}}})}
\over 2 (C^{[\mathfrak{p}]}(t)- t^{n_1^{\mathfrak{p}}})}.$$
\end{frame}



%\begin{frame}\frametitle{Theorem 1 -b-}
%... where  $\delta_{i,j}$ is the Kronecker delta,
%$$\sum_{i = 0}^{n_1^{\mathfrak{p}}-1} \alpha_{i,0}t^{i}=\sum_{i = 0}^{n_1^{\mathfrak{p}}-1} c_{2i}t^{i}-
%\delta_{-1,n_0^{\mathfrak{p}}-n_1^{\mathfrak{p}}}t^{n_1^{\mathfrak{p}}-1}, $$
%$$\sum_{i = 0}^{n_1^{\mathfrak{p}}-1} \alpha_{i,1}t^{i}=-\sum_{i = 0}^{n_1^{\mathfrak{p}}-1} c_{2(i+1)}t^{i}-
%\delta_{0,n_0^{\mathfrak{p}}-n_1^{\mathfrak{p}}}t^{n_1^{\mathfrak{p}}-1}, $$
%$$\sum_{i = 0}^{n_1^{\mathfrak{p}}-1} \alpha_{i,2}t^{i}=\sum_{i = 0}^{n_1^{\mathfrak{p}}-1} c_{2(i+1)}t^{i}-
%\delta_{1,n_0^{\mathfrak{p}}-n_1^{\mathfrak{p}}}t^{n_1^{\mathfrak{p}}-1},$$ and the coefficients $c_i$ are given by the autocorrelation vector
%of $\mathfrak{p}.$
% An analogous formula holds for $h^{[\bar{\mathfrak{p}}]}(t)$.
%\end{frame}



\begin{frame}\frametitle{Formulae for classes of patterns}
{\small
\begin{itemize}
\item $\mathfrak{p}=1^{j+1}0^j$
$$ d^{[\mathfrak{p}]}(t)={1 \over \sqrt{1-4t+4t^{j+1}}}, \quad h^{[\mathfrak{p}]}(t)={1 -\sqrt{1-4t+4t^{j+1}} \over 2 }$$
\item $\mathfrak{p}=0^{j+1}1^j$
$$ d^{[\mathfrak{p}]}(t)={1 \over \sqrt{1-4t+4t^{j+1}}}, \quad h^{[\mathfrak{p}]}(t)={1 -\sqrt{1-4t+4t^{j+1}} \over 2(1-t^j) }$$
\item $\mathfrak{p}=1^{j}0^j$ and $\mathfrak{p}=0^{j}1^j$
$$ d^{[\mathfrak{p}]}(t)={1 \over \sqrt{1-4t+2t^j+t^{2j}}}, \quad h^{[\mathfrak{p}]}(t)={{1+t^j -\sqrt{1-4t+2t^j+t^{2j}} } \over 2 }$$
\end{itemize}
}
\end{frame}

\begin{frame}\frametitle{Formulae for classes of patterns}
{\small
\begin{itemize}
\item $\mathfrak{p}=(10)^j1$
\begin{displaymath}
\begin{split}
d^{[\mathfrak{p}]}(t)&={
\sum_{i=0}^j t^i \over \sqrt{1-2\sum_{i=1}^jt^i-3 \left(
\sum_{i=1}^j t^i \right)^2}}, \\
h^{[\mathfrak{p}]}(t)&={\sum_{i=0}^jt^i
-\sqrt{1-2\sum_{i=1}^jt^i-3\left( \sum_{i=1}^j t^i \right)^2}
\over 2 \sum_{i=0}^jt^i}
\end{split}
\end{displaymath}
\item
$\mathfrak{p}=(01)^j0$
\begin{displaymath}
\begin{split}
d^{[\mathfrak{p}]}(t)&={
\sum_{i=0}^j t^i \over \sqrt{1-2\sum_{i=1}^jt^i-3 \left( \sum_{i=1}^j t^i \right)^2}}, \\
h^{[\mathfrak{p}]}(t)&={\sum_{i=0}^jt^i -\sqrt{1-2\sum_{i=1}^jt^i-3\left( \sum_{i=1}^j t^i \right)^2} \over 2 \sum_{i=0}^{j-1}t^i}
\end{split}
\end{displaymath}

\end{itemize}
}
\end{frame}

\begin{frame}\frametitle{A combinatorial interpretation for $\mathfrak{p}=10$}

In this case we get the RA ${\mathcal{R}^{[10]}} = \left(d^{[10]}(t), h^{[10]}(t)\right)$ 
such that
\begin{displaymath} 
d^{[10]}(t)=\frac{1}{1-t} \quad \text{and} \quad h^{[10]}(t) = t, 
\end{displaymath} so the number $R_{n, 0}^{[10]}$ of words
containing $n$ bits $1$ and $n$ bits $0$, avoiding pattern $\mathfrak{p}=10$, is
$[t^{n}] d^{[10]}(t) = 1$ for $n\in\mathbb{N}$. 

In terms of lattice paths this corresponds to the fact that there is exactly
one \emph{valley}-shaped path having $n$ steps of both kinds $\diagup$ and
$\diagdown$, avoiding $\mathfrak{p}=10$ and terminating at coordinate $(2n, 0)$
for each $n\in\mathbb{N}$, formally the path $0^{n}1^{n}$.

\end{frame}


\iffalse
\begin{frame}\frametitle{A Lemma}

 Let $\mathfrak{p}$ be  a Riordan pattern. Then the Riordan array ${{R}^{[\mathfrak{p}]}}$ is characterized by the  $A$-matrix
defined by the following relation:
$$R_{n+1,k+1}^{[\mathfrak{p}]}=R_{n,k}^{[\mathfrak{p}]}
+R_{n+1,k+2}^{[\mathfrak{p}]}-R_{n+1-n_1^{\mathfrak{p}},k+1+n_0^{\mathfrak{p}}-n_1^{\mathfrak{p}}}^{[\mathfrak{p}]} +$$
$$- \sum_{i\geq 1} c_{2i}\left(  R_{n+1-i,k+1}^{[\mathfrak{p}]} -R_{n-i,k}^{[\mathfrak{p}]} -R_{n+1-i,k+2}^{[\mathfrak{p}]} \right),$$
where  the  $c_i$ are given by the autocorrelation vector of
$\mathfrak{p}.$
\end{frame}
\fi

\section{The $|w|_{0}\leq |w|_{1}$ constraint}

\begin{frame}\frametitle{The $|w|_{0}\leq |w|_{1}$ constraint}
\begin{itemize}
\item let $|w|_{i}$ be the number of bits $i$ in word $w$
\item enumeration of binary words avoiding a pattern $\mathfrak{p}$, without the
constraint $|w|_0\leq |w|_1,$ gives a rational bivariate generating function
for the sequence $F^{[\mathfrak{p}]}_n=\sum_{k=0}^nF_{n,k}^{[\mathfrak{p}]}$
\item under the restriction such that words have to have no more bits $0$ than
bits $1$, then the language is no longer regular and its enumeration becomes
more difficult
\item using gf $R^{[\mathfrak{p}]}(x,y)$ and the fundamental theorem of RAs:
$$\sum_{k=0}^n d_{n,k}f_k=[t^n]d(t)f(h(t)) $$ we obtain many {\bf \red new
algebraic generating functions} expressed in terms of the autocorrelation
polynomial of  $\mathfrak{p}$
\end{itemize}
\end{frame}

\begin{frame}\frametitle{Theorem 2: the case $n_1^{[\mathfrak{p}]}-n_0^{[\mathfrak{p}]}=1$}
Recall that 
\begin{displaymath}
    R^{[\mathfrak{p}]}(t,w)=\sum_{n,k\in\mathbb{N}} R_{n, k}^{[\mathfrak{p}]}t^n
    w^k={d^{[\mathfrak{p}]}(t) \over 1-wh^{[\mathfrak{p}]}(t)}
\end{displaymath}
Let $S^{[\mathfrak{p}]}(t)=\sum_{n\geq 0}S_n^{[\mathfrak{p}]}t^n$ be the gf
enumerating the set of binary words $\left\lbrace
w\in\mathcal{L}^{[\mathfrak{p}]} : |w|_0\leq |w|_1\right\rbrace$ according to
the number of bits $1$
\begin{itemize}
\item if $n_1^{[\mathfrak{p}]}=n_0^{[\mathfrak{p}]}+1:$
$$S^{[\mathfrak{p}]}(t)={2C^{[\mathfrak{p}]}(t) \over \sqrt{Q(t)}\left(\sqrt{C^{[\mathfrak{p}]}(t)}+ \sqrt{Q(t)} \right)} $$
    where $Q(t)={(1-4t)C^{[\mathfrak{p}]}(t)^2+4t^{n_1^{[\mathfrak{p}]}}}$
\end{itemize}
\end{frame}

\begin{frame}\frametitle{Theorem 2: the case $n_0^{[\mathfrak{p}]}-n_1^{[\mathfrak{p}]}=1$}
\begin{itemize}
\item if $n_0^{[\mathfrak{p}]}=n_1^{[\mathfrak{p}]}+1:$
$$S^{[\mathfrak{p}]}(t)={2C^{[\mathfrak{p}]}(t)(C^{[\mathfrak{p}]}(t)-t^{n_1^{[\mathfrak{p}]}}
) \over \sqrt{Q(t)} \left(C^{[\mathfrak{p}]}(t)-2t^{n_1^{[\mathfrak{p}]}}+ \sqrt{Q(t)} \right) }$$
    where $Q(t)={ (1-4t)C^{[\mathfrak{p}]}(t)^2+4t^{n_0^{[\mathfrak{p}]}}C^{[\mathfrak{p}]}(t)}$
\end{itemize}
\end{frame}

\begin{frame}\frametitle{Theorem 2: the case $n_0^{[\mathfrak{p}]}-n_1^{[\mathfrak{p}]}=0$}
\begin{itemize}
\item if $n_1^{[\mathfrak{p}]}=n_0^{[\mathfrak{p}]}:$
$$S^{[\mathfrak{p}]}(t)={2C^{[\mathfrak{p}]}(t)^2 \over \sqrt{Q(t)}
    \left(C^{[\mathfrak{p}]}(t)-t^{n_0^{[\mathfrak{p}]}}+ \sqrt{Q(t)} \right) }$$
where $Q(t)=(1-4t)C^{[\mathfrak{p}]}(t)^2+2t^{n_0^{[\mathfrak{p}]}}C^{[\mathfrak{p}]}(t)+t^{2n_0^{[\mathfrak{p}]}}$
\end{itemize}

\begin{proof}
Observe that $S^{[\mathfrak{p}]}(t)=R^{[\mathfrak{p}]}(t,1),$ or, equivalently, that
$S_n^{[\mathfrak{p}]}=\sum_{k=0}^nR_{n, k}^{[\mathfrak{p}]}$ and apply the
fundamental rule with $f_k=1$.
\end{proof}
\end{frame}


\begin{frame}\frametitle{Theorem 3: the case $n_1^{[\mathfrak{p}]}-n_0^{[\mathfrak{p}]}=1$}
Let $L^{[\mathfrak{p}]}(t)=\sum_{n\geq 0}L_n^{[\mathfrak{p}]}t^n$ be the
gf enumerating the set of binary words $\left\lbrace
w\in\mathcal{L}^{[\mathfrak{p}]} : |w|_0\leq |w|_1\right\rbrace$ according to
the length

\begin{itemize}
\item if $n_1^{[\mathfrak{p}]}=n_0^{[\mathfrak{p}]}+1:$
$$L^{[\mathfrak{p}]}(t)= {2tC^{[\mathfrak{p}]}(t^2)^2 \over \sqrt{Q(t)}\left((2t-1)C(t^2)+ \sqrt{ Q(t) } \right)}$$
where $Q(t)=C^{[\mathfrak{p}]}(t^2)\left( (1-4t^2)C^{[\mathfrak{p}]}(t^2)+4t^{2n_1^{[\mathfrak{p}]}}\right)$
\end{itemize}
\end{frame}

\begin{frame}\frametitle{Theorem 3: the case $n_0^{[\mathfrak{p}]}-n_1^{[\mathfrak{p}]}=1$}
\begin{itemize}
\item if $n_0^{[\mathfrak{p}]}=n_1^{[\mathfrak{p}]}+1:$
$$L^{[\mathfrak{p}]}(t)={2t\sqrt{C^{[\mathfrak{p}]}(t^2)}(t^{2n_1^{[\mathfrak{p}]}}-C^{[\mathfrak{p}]}(t^2))
\over \sqrt{ Q(t) }\left((1-2t)C^{[\mathfrak{p}]}(t^2)+ B(t) - \sqrt{C^{[\mathfrak{p}]}(t^2) Q(t) } \right)}$$
where $Q(t)=(1-4t^2)C^{[\mathfrak{p}]}(t^2)+4t^{2n_0^{[\mathfrak{p}]}}$ and
$B(t)=2t^{n_0^{[\mathfrak{p}]} +n_1^{[\mathfrak{p}]}}$
\end{itemize}
\end{frame}

\begin{frame}\frametitle{Theorem 3: the case $n_1^{[\mathfrak{p}]}-n_0^{[\mathfrak{p}]}=0$}
\begin{itemize}
\item if $n_1^{[\mathfrak{p}]}=n_0^{[\mathfrak{p}]}:$ $$L^{[\mathfrak{p}]}(t)=
{2tC^{[\mathfrak{p}]}(t^2)^2 \over \sqrt{ Q(t)
}\left((2t-1)C(t^2)-t^{2n_0^{[\mathfrak{p}]}} + \sqrt{ Q(t) } \right)}$$
where
$Q(t)=(1-4t^2)C^{[\mathfrak{p}]}(t^2)^2+2t^{2n_0^{[\mathfrak{p}]}}C^{[\mathfrak{p}]}(t^2)+t^{4n_0^{[\mathfrak{p}]}}$
\end{itemize}
\end{frame}

\begin{frame}\frametitle{Theorem 3: proof}
\begin{proof}
Observe that the application of generating function $R^{[\mathfrak{p}]}(t, w)$ as
\begin{displaymath}
 R^{[\mathfrak{p}]}\left(tw,{1 \over w}\right)=\sum_{n,k\in\mathbb{N}} R_{n, k}^{[\mathfrak{p}]}t^n w^{n-k}
\end{displaymath}
entails that $[t^{r}w^{s}]R^{[\mathfrak{p}]}\left(tw,{1 \over w}\right)=R_{r,
r-s}^{[\mathfrak{p}]}$ which is the number of binary words with $r$ bits $1$
and $s$ bits $0$. To enumerate according to the length let $t=w$, therefore
$$L^{[\mathfrak{p}]}(t)=\sum_{n\geq 0}L_n^{[\mathfrak{p}]}t^n=R^{[\mathfrak{p}]}\left(t^2,\frac{1}{t}\right)$$
\end{proof}
\end{frame}


\section{Series developments and closed formulae}

\begin{frame}\frametitle{Formulae for classes of patterns}
{\small
\begin{itemize}

\item for $\mathfrak{p}=1^{j+1}0^j$ we have:
$$ S^{[\mathfrak{p}]}(t)={2 \over \sqrt{Q(t)}\left(1+ \sqrt{Q(t)}\right) }, \quad Q(t)=1-4t+4t^{j+1}$$

\item for $\mathfrak{p}=0^{j+1}1^j$ we have:
$$ S^{[\mathfrak{p}]}(t)={2(1-t^j) \over \sqrt{Q(t)} \left(1-2t^j+ \sqrt{Q(t)}\right)}, \quad Q(t)=1-4t+4t^{j+1}$$

\item for $\mathfrak{p}=1^{j}0^j$ and $\mathfrak{p}=0^{j}1^j$ we have:
$$ S^{[\mathfrak{p}]}(t)={2 \over \sqrt{Q(t)} \left(1-t^j+\sqrt{Q(t)}  \right)}, \quad Q(t)=1-4t+2t^j+t^{2j}$$

\end{itemize}
}
\end{frame}

\begin{frame}\frametitle{Formulae for classes of patterns}
{\small
\begin{itemize}

\item for $\mathfrak{p}=(10)^j1$ we have:
$$ S^{[\mathfrak{p}]}(t)={2 (1-t^{j+1})\over 1-4t+3t^{j+1}+\sqrt{Q(t)}}$$
where $Q(t)=1-4t+2t^{j+1}+4t^{j+2}-3t^{2j+2}$

\item for $\mathfrak{p}=(01)^j0$ we have:
$$ S^{[\mathfrak{p}]}(t)={2 (1-t^j-t^{j+1}+t^{2j+1})\over \sqrt{Q(t)} \left(1-2t^j+t^{j+1}+\sqrt{Q(t)}  \right)}$$
where $Q(t)={1-4t+2t^{j+1}+4t^{j+2}-3t^{2j+2}}$

\end{itemize}
}
\end{frame}

\begin{frame}\frametitle{Series development for $S^{[1^{j+1}0^{j}]}(t)$}
{\tiny
\begin{table}
\begin{equation*}\begin{array}{c|cccccccccccc}j/n & 0 & 1 & 2 & 3 & 4 & 5 & 6 & 7 & 8 & 9 & 10 & 11\\\hline0 & 1 & 0 & 0 & 0 & 0 & 0 & 0 & 0 & 0 & 0 & 0 & 0\\1 & 1 & 3 & 7 & 15 & 31 & 63 & 127 & 255 & 511 & 1023 & 2047 & 4095\\2 & 1 & 3 & 10 & 32 & 106 & 357 & 1222 & 4230 & 14770 & 51918 & 183472 & 651191\\3 & 1 & 3 & 10 & 35 & 123 & 442 & 1611 & 5931 & 22010 & 82187 & 308427 & 1162218\\4 & 1 & 3 & 10 & 35 & 126 & 459 & 1696 & 6330 & 23806 & 90068 & 342430 & 1307138\\5 & 1 & 3 & 10 & 35 & 126 & 462 & 1713 & 6415 & 24205 & 91874 & 350406 & 1341782\\6 & 1 & 3 & 10 & 35 & 126 & 462 & 1716 & 6432 & 24290 & 92273 & 352212 & 1349768\\7 & 1 & 3 & 10 & 35 & 126 & 462 & 1716 & 6435 & 24307 & 92358 & 352611 & 1351574\\8 & 1 & 3 & 10 & 35 & 126 & 462 & 1716 & 6435 & 24310 & 92375 & 352696 & 1351973\end{array}\end{equation*}
\begin{displaymath}
\begin{split}
[t^{3}]S^{[110]}(t) &= \big|\lbrace 111, 0111, 1011, 00111, 01011, 10011, 10101, 000111, \\
& 001011, 010011, 010101, 100011, 100101, 101001, 101010\rbrace\big| = 15
\end{split}
\end{displaymath}
\caption{Some series developments for $S^{[1^{j+1}0^j]}(t)$ and the set of
words with $n=3$ bits $1$, avoiding pattern $\mathfrak{p}=110$, so $j=1$ in the
family; moreover, for $j=1$ the sequence corresponds to $A000225$, for $j=2$ the
sequence corresponds to $A261058$.}
\end{table}
}
\end{frame}

\begin{frame}\frametitle{formulae for classes of patterns}
{\small
\begin{itemize}

\item for $\mathfrak{p}=1^{j+1}0^j$ we have:
$$ L^{[\mathfrak{p}]}(t)={2t \over \sqrt{Q(t)}\left(2t-1+ \sqrt{Q(t)}\right) }, \quad Q(t)=1-4t^2+4t^{2(j+1)}$$

\item for $\mathfrak{p}=0^{j+1}1^j$ we have:
$$ L^{[\mathfrak{p}]}(t)={2t(t^{2j}-1) \over \sqrt{Q(t)} \left(1-2t+2t^{2j+1} -\sqrt{Q(t)}\right)}, \quad Q(t)=1-4t^2+4t^{2(j+1)}$$

\item for $\mathfrak{p}=1^{j}0^j$ and $\mathfrak{p}=0^{j}1^j$ we have:
$$ L^{[\mathfrak{p}]}(t)={2 t \over \sqrt{Q(t)} \left(-1+2t-t^{2j}+\sqrt{Q(t)}  \right)}, \quad Q(t)=1-4t^{2}+2t^{2j}+t^{4j}$$

\end{itemize}
}
\end{frame}

\begin{frame}\frametitle{formulae for classes of patterns}
{\small
\begin{itemize}

\item for $\mathfrak{p}=(10)^j1$ we have:
$$ L^{[\mathfrak{p}]}(t)={2 t (t^{2j+2}-1)\over 1-4t^2+3t^{2j+2}+(2t-1)\sqrt{Q(t)} }$$
where $Q(t)=1-4t^2+2t^{2j+2}+4t^{2j+4}-3t^{4j+4}$

\item for $\mathfrak{p}=(01)^j0$ we have:
$$ L^{[\mathfrak{p}]}(t)={2 t (t^{2j+2}-1)(t^{2j}-1)\over  \sqrt{Q(t)} \left(t^{2j+2}-2t^{2j+1}+2t-1+ \sqrt{Q(t)}\right) }$$
where $Q(t)={1-4t^2+2t^{2j+2}+4t^{2j+4}-3t^{4j+4}}$

\end{itemize}
}
\end{frame}

\begin{frame}\frametitle{Series development for $L^{[1^{j+1}0^{j}]}(t)$}
{\tiny
\begin{table}
\begin{equation*}\begin{array}{c|ccccccccccccccc}j/n & 0 & 1 & 2 & 3 & 4 & 5 & 6 & 7 & 8 & 9 & 10 & 11 & 12 & 13 & 14\\\hline0 & 1 & 0 & 0 & 0 & 0 & 0 & 0 & 0 & 0 & 0 & 0 & 0 & 0 & 0 & 0\\1 & 1 & 1 & 3 & 3 & 7 & 7 & 15 & 15 & 31 & 31 & 63 & 63 & 127 & 127 & 255\\2 & 1 & 1 & 3 & 4 & 11 & 15 & 38 & 55 & 135 & 201 & 483 & 736 & 1742 & 2699 & 6313\\3 & 1 & 1 & 3 & 4 & 11 & 16 & 42 & 63 & 159 & 247 & 610 & 969 & 2354 & 3802 & 9117\\4 & 1 & 1 & 3 & 4 & 11 & 16 & 42 & 64 & 163 & 255 & 634 & 1015 & 2482 & 4041 & 9752\\5 & 1 & 1 & 3 & 4 & 11 & 16 & 42 & 64 & 163 & 256 & 638 & 1023 & 2506 & 4087 & 9880\\6 & 1 & 1 & 3 & 4 & 11 & 16 & 42 & 64 & 163 & 256 & 638 & 1024 & 2510 & 4095 & 9904\\7 & 1 & 1 & 3 & 4 & 11 & 16 & 42 & 64 & 163 & 256 & 638 & 1024 & 2510 & 4096 & 9908\end{array}\end{equation*}
\caption{Some series developments for $L^{[1^{j+1}0^j]}(t)$; moreover, for
$j=1$ the sequence corresponds to $A052551$.}
\end{table}
}
\end{frame}
\begin{frame}\frametitle{Closed formulae for particular cases}

When the parameter $j$ for a pattern $\mathfrak{p}$ assumes values $0$ and $1$
it is possible to find closed formulae for coefficients
$S_{n}^{[\mathfrak{p}]}$ and $L_{n}^{[\mathfrak{p}]}$; moreover, in a recent
submitted paper we give combinatorial interpretations, in terms of inversions
in words and boxes occupancy, too.

\begin{block}{$S_{n}^{[\mathfrak{p}]}$}
\begin{displaymath}
\begin{array}{c|ccc}
j/\mathfrak{p} & {1^{j+1}0^{j}} & {0^{j+1}1^{j}} & {1^{j}0^{j}} \\
\hline
0 &  [\![n = 0]\!] &  1 & { {2n+1}\choose{n} } \\
1 &  2^{n+1} -1 &  (n+2)2^{n-1} & n+1 \\
\end{array}{}
\end{displaymath}
\end{block}
\end{frame}

\begin{frame}\frametitle{Closed formulae for particular cases}
\begin{block}{$L_{2m}^{[\mathfrak{p}]}$}
\begin{displaymath}
\begin{array}{c|ccc}
j/\mathfrak{p} & {1^{j+1}0^{j}} & {0^{j+1}1^{j}} & {1^{j}0^{j}} \\
\hline
0 &  [\![n = 0]\!] &  1 & 2^{2m-1} + \frac{1}{2}{ {2m}\choose{m} } \\
1 &  2^{m+1} -1 &  F_{2m+3}-2^{m} & m+1 \\
\end{array}{}
\end{displaymath}
\end{block}

\begin{block}{$L_{2m+1}^{[\mathfrak{p}]}$}
\begin{displaymath}
\begin{array}{c|ccc}
j/\mathfrak{p} & {1^{j+1}0^{j}} & {0^{j+1}1^{j}} & {1^{j}0^{j}} \\
\hline
0 &  0 &  1 & 2^{2m-1} \\
1 &  2^{m+1} -1 & F_{2m+3}-2^{m+1} & m+1 \\
\end{array}{}
\end{displaymath}
\end{block}
\end{frame}

\section*{Summary}

\begin{frame}{Summary}

  % Keep the summary *very short*.
  \begin{block}{Key points}
      \begin{itemize}
      \item split $F(t)$ in $F^{[\mathfrak{p}]}(x,y)$ to account for bits $1$ and $0$
      \item ${{R}^{[\mathfrak{p}]}}$ and ${{R}^{[\bar{\mathfrak{p}]}}}$ are both RA
      $\leftrightarrow$  $\mathfrak{p}$ is a Riordan pattern.
      \item requiring $|w|_{0}\leq|w|_{1}$ entails
          \begin{displaymath}
          \begin{split}
          S^{[\mathfrak{p}]}(t)&=R^{[\mathfrak{p}]}(t,1)\rightarrow 
            [t^{n}]S^{[\mathfrak{p}]}(t)= \left|\left\lbrace w \in  \mathcal{L}^{[\mathfrak{p}]}:
                \begin{array}{l} |w|_{1} = n \\ |w|_{0}\leq|w|_{1} \end{array}\right\rbrace\right|\\
          L^{[\mathfrak{p}]}(t)&=R^{[\mathfrak{p}]}\left(t^2,\frac{1}{t}\right)\rightarrow
            [t^{n}]L^{[\mathfrak{p}]}(t)= \left|\left\lbrace w \in  \mathcal{L}^{[\mathfrak{p}]}:
                \begin{array}{l} |w| = n \\ |w|_{0}\leq|w|_{1} \end{array}\right\rbrace\right|\\
          \end{split}
          \end{displaymath}
      \end{itemize}
  \end{block}
\end{frame}
  
\begin{frame}{Outlook}
  % The following outlook is optional.
  \vskip0pt plus.5fill
    \begin{itemize}
    \item provide combinatorial interpretations for both pattern classes $(10)^{j}1$ and
    $(01)^{j}0$, at least for $j\in \lbrace 0,1 \rbrace$
    \item conjecture: when $j > 1$ in pattern classes it seems that $R^{[\mathfrak{p}]}$ is
    a binomial transformation
    \item build the Riordan graph for both RAs $R^{[\mathfrak{p}]}$ and $R^{[\bar{\mathfrak{p}}]}$
        to study the meaning of pattern avoidance at graph level
    \end{itemize}
\end{frame}

\begin{frame}{ }
\begin{CJK}{UTF8}{mj}
\Huge 고맙습니다
\end{CJK}
\end{frame}

\end{document}


