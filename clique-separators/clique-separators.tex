% $Header$

\documentclass{beamer}

% This file is a solution template for:

% - Giving a talk on some subject.
% - The talk is between 15min and 45min long.
% - Style is ornate.



% Copyright 2004 by Till Tantau <tantau@users.sourceforge.net>.
%
% In principle, this file can be redistributed and/or modified under
% the terms of the GNU Public License, version 2.
%
% However, this file is supposed to be a template to be modified
% for your own needs. For this reason, if you use this file as a
% template and not specifically distribute it as part of a another
% package/program, I grant the extra permission to freely copy and
% modify this file as you see fit and even to delete this copyright
% notice. 


\mode<presentation>
{
  \usetheme{Pittsburgh}
  \usecolortheme{dove}
  \usefonttheme{professionalfonts}
  \setbeamertemplate{blocks}[rounded][shadow=false]
  \setbeamercolor{block title}{fg=structure,bg=gray}
}


\usepackage[english]{babel}
\usepackage[utf8]{inputenc}
\usepackage[T1]{fontenc}
\usepackage{CJKutf8}
\usepackage{amsfonts}
\usepackage{color}
\usepackage{pstricks,pst-text}
\usepackage{pst-node,pst-tree}
\usepackage{concrete}

\usepackage{amsmath}%
\usepackage{amsthm}
\usepackage{amssymb}%
\usepackage{wasysym}%
\usepackage{graphicx}
\usepackage{float}
\usepackage{caption}
\usepackage{bbold}
% Or whatever. Note that the encoding and the font should match. If T1
% does not look nice, try deleting the line with the fontenc.


\title{Decomposition by clique separators}

%\subtitle {Presentation Subtitle} % (optional)

\author[Merlini, Nocentini] % (optional, use only with lots of authors)
{Massimo Nocentini}
% - Use the \inst{?} command only if the authors have different
%   affiliation.

\institute
{
  Dipartimento di Statistica, Informatica, Applicazioni \\
  University of Florence, Italy
}
% - Use the \inst command only if there are several affiliations.
% - Keep it simple, no one is interested in your street address.

\date[Short Occasion] % (optional)
{\today}

\subject{
This talk has an educational purpose to understand concepts explored in the
paper "Decomposition by clique separators" by Robert E. Tarjan; moreover, we
use it to pass a PhD course taught by prof. Giovanni Marchetti @ University of
Florence.  Tarjan considers the problem of decomposing a graph by means of clique
separators, by which he means finding cliques (complete graphs) whose removal
disconnects the graph. 
}
% This is only inserted into the PDF information catalog. Can be left
% out. 



% If you have a file called "university-logo-filename.xxx", where xxx
% is a graphic format that can be processed by latex or pdflatex,
% resp., then you can add a logo as follows:

% \pgfdeclareimage[height=0.5cm]{university-logo}{university-logo-filename}
% \logo{\pgfuseimage{university-logo}}



% Delete this, if you do not want the table of contents to pop up at
% the beginning of each subsection:
\AtBeginSection[]
{
  \begin{frame}<beamer>{Outline}
    \tableofcontents[currentsection]
  \end{frame}
}


% If you wish to uncover everything in a step-wise fashion, uncomment
% the following command: 

%\beamerdefaultoverlayspecification{<+->}


\begin{document}

\begin{frame}
  \titlepage
\end{frame}

\iffalse
\begin{frame}{Outline}
  \tableofcontents
  % You might wish to add the option [pausesections]
\end{frame}
\fi


% Since this a solution template for a generic talk, very little can
% be said about how it should be structured. However, the talk length
% of between 15min and 45min and the theme suggest that you stick to
% the following rules:  

% - Exactly two or three sections (other than the summary).
% - At *most* three subsections per section.
% - Talk about 30s to 2min per frame. So there should be between about
%   15 and 30 frames, all told.


\begin{frame}{intro}

\begin{block}{we aim to}
\begin{itemize}
    \item study the paper \textit{Decomposition by clique
    separators}\\{\footnotesize by Robert E. Tarjan, Discrete Mathematics, Volume 55,
    Issue 2, July 1985}
    \item understand the fundamental idea of \textit{vertex elimination order}
    \item write a Python one-liner
\end{itemize}
\end{block}

\begin{block}{the big picture}
Tarjan considers the problem of decomposing a graph by means of \textbf{clique
separators}, by which he means finding \textit{cliques (complete graphs) whose removal
disconnects the graph.} 
\end{block}

\end{frame}


\begin{frame}{defs}
$G = (V, E)$ denotes a \textit{connected, undirected} graph where
\begin{itemize}
    \item $V$ is a set of vertices $\lbrace v_{0}, \ldots, v_{n-1} \rbrace$
    \item $E$ is a set of edges $\lbrace e_{0}, \ldots, e_{m-1} \rbrace$ \\
    such that $e_{i}= \lbrace u, v \rbrace$ for some $u, v\in V$
\end{itemize}
\vfill
$G(X)=\left(X,  \lbrace \lbrace u, v \rbrace \in E: u, v\in X \rbrace\right)$ 
\begin{itemize}
    \item provided that $G=(V, E)$ and $X\subseteq V$
    \item $G(X)$ is the subgraph of $G$ induced by vertex set $X$
\end{itemize}
\vfill
$G(V\setminus X)$ is disconnected $\rightarrow$ $X$ is a \textit{separator} for $G$
\vfill
$C= \lbrace c_{0}, \ldots, c_{h} \rbrace\subseteq V$ is a \textit{clique}
$\leftrightarrow$ $\lbrace c_{i}, c_{j} \rbrace \in E$ for all $c_{i}\neq c_{j}\in C$
\end{frame}

\begin{frame}{defs}
let $\alpha = (v_{0}, v_{1}, \ldots, v_{k-1}, v_{0})$ be a cycle of length $k$
\vfill
$\alpha$ has a \textit{chord} if $\lbrace v_{i},v_{i+j} \rbrace\in E$ where
$j>1$, for some $i<k$
\vfill
$G=(V, E)$ is \textit{chordal} if every cycle $\alpha$ of length $k\geq 4$ has a chord
\vfill
a bijection $\pi: V \rightarrow  \lbrace 0, \ldots, n-1 \rbrace$  is an
\textit{elimination ordering} for $G$
\vfill
let $F_{\pi} = \left\lbrace \lbrace u, v \rbrace \not\in E:
        \begin{array}{l}
            (u, w_{0}, \ldots, w_{k-1}, v) \text{ is a path in } G\\ 
            \text{such that } \pi(w_{j})< \min{ \lbrace \pi(u), \pi(v) \rbrace}\\
            \text{where } j\in  \lbrace 0, \ldots, k-1 \rbrace \wedge u\neq v
        \end{array}
        \right\rbrace$ in:
\begin{itemize}
    \item $F_{\pi}$ is a \textit{fill-in edge set} for $\pi$
    \item $G_{\pi}=(V, E\cup F_{\pi})$ is a \textit{fill-in graph} for $\pi$ 
    \item $\pi$ is \textit{perfect} if $F_{\pi}=\emptyset$
    \item $\pi$ is \textit{minimal} if no ordering $\rho$ exists such that
    $F_{\rho}\subset F_{\pi}$
\end{itemize}
\end{frame}

\begin{frame}{approach}
\begin{itemize}
    \item \textit{divide and conquer}: decompose $G$ into a hierarchy of
    components; then, solve the problem in each component at the bottom;
    finally, piece together solutions all the way up for the whole $G$
    \item decompositions exists already, \textit{respect to the size} of the
    separator set $X$.  On the contrary, Tarjan studies \textit{respect to the
    structure}, namely $X$ has to be a \textit{clique}
    \item assume $G=(V, E)$ has a clique separator $C$, therefore $ \lbrace A,
    B, C \rbrace$ partitions $V$. So, $G$ decomposes to $G_{AC}=G(A\cup C)$ and
    $G_{BC}=G(B\cup C)$. Since they are both graphs, then recurs on both $G_{AC}$
    and $G_{BC}$ till no further decomposition is possible
\end{itemize}
\end{frame}

\begin{frame}{thms}
\begin{theorem}
$\pi$ elimination ordering for $G$ is perfect $\leftrightarrow$ $G$ is chordal
\end{theorem}
\begin{proof}
$(\leftarrow)$ By hp, every cycle $(u, w_{0}, \ldots, w_{k-1}, v, u)$ in $G$ has a chord, 
so for each smallest cycle $(u, w, z, v)$ such that
$\pi(w),\pi(z)<\min{\lbrace\pi(u), \pi(v)\rbrace}$ exists either $ \lbrace u,z
\rbrace\in E$ or $ \lbrace w,v \rbrace\in E$, hence $F_{\pi}=\emptyset$.
\vfill
$(\rightarrow)$ By hp, $G=G_{\pi}$ because $F_{\pi}=\emptyset$, so for every
path $(u, w_{0}, \ldots, w_{k-1}, v)$ such that $\pi(w_{j})< \min{ \lbrace \pi(u), \pi(v) \rbrace}$
exists $ \lbrace u, v \rbrace\in E$. Hence exists $ \lbrace u, w_{i} \rbrace\in
E$ chord for cycle $(u, w_{0}, \ldots, w_{k-1}, v, u)$, making $G$ chordal.

\end{proof}
\end{frame}

\begin{frame}{thms}
\begin{theorem}
let $\pi$ elimination ordering for $G$ be minimal in: 
\begin{center}
$C$ is a clique separator for $G$ $\rightarrow$ $CC_{G(V\setminus C)}(u) =
CC_{G(V\setminus C)}(v)$
\end{center}
for all $ \lbrace u, v \rbrace \in F_{\pi}$ and $CC_{G}(w)= \lbrace z\in V:
    (w,\ldots,z) \text{\small\,is a path in } G \rbrace$
\end{theorem}
\begin{proof}
By contraposition, exists $ \lbrace u, v \rbrace\in F_{\pi}$ such that
\begin{center}
$C$ \textit{is not} a clique separator for $G$ $\leftarrow$ $CC_{G(V\setminus C)}(u)\neq 
CC_{G(V\setminus C)}(v)$
\end{center}
Assume not, so let $C= \lbrace a, b, v \rbrace$ be a clique separator for $G$,
where $a, b\in CC_{G(V\setminus C)}(u)$ and let $ \lbrace a, u \rbrace\in E$, wlog.
$(u, a, b, v, u)$ is a smallest cycle in $G$ through $C$ with chord $ \lbrace a, v \rbrace$, 
therefore $G$ is chordal, hence $G=G_{\pi}$ so $F_{\pi}=\emptyset \not\ni \lbrace u, v \rbrace$,
a contraddiction.
\end{proof}
\end{frame}

\begin{frame}{decomposition}
compute $\pi$ elimination ordering for $G=(V, E)$
\vfill
let $C_{\pi}(v) =  \lbrace w \in V : \pi(v) < \pi(w) \wedge  \lbrace v, w \rbrace \in E\cup F_{\pi} \rbrace$ in
$$D(G, v) = \left\lbrace\begin{array}{ll}
            G(B_{\pi}(v)\cup C_{\pi}(v)) & \text{if } C_{\pi}(v) \text{ is a clique}\\
            G & \text{otherwise} \\
           \end{array}\right.$$
where $B_{\pi}(v) = V\setminus \left(C_{\pi}(v)\cup A_{\pi}(v)\right)$
and $A_{\pi}(v) = CC_{G\left(V\setminus C_{\pi}(v)\right)}(v)$.
\vfill
%if $D(G, v)$ returns when $C_{\pi}(v)$ is a clique, then $C_{\pi}(v)$ is a \emph{clique separator}
if $D(G, v) = G$ then  $C_{\pi}(v)$ is a \emph{clique separator},\\
so decompose $G$ into $G(A_{\pi}(v) \cup C_{\pi}(v))$ and $G(B_{\pi}(v) \cup C_{\pi}(v))$.
\vfill
\begin{block}{Python code}
\textit{functools.reduce}(D, \textit{sorted}($V$, key=$\pi$), $G$)
\end{block}
\end{frame}

\begin{frame}{facts}
the algorithm is both \textit{correct} and \textit{complete}
\vfill
it has complexity $O(nm)$, where $n=|V|$ and $m=|E|$:
\begin{itemize}
    \item computing $\pi$ minimal ordering requires $O(nm)$
    \item computing $C_{\pi}$ reduces to $F_{\pi}$, which requires $O(n^{2})$
    \item folding for each $v\in V$ requires $O(m)$
\end{itemize}
\vfill
the binary tree representing cliques and decompositions is as \textit{skewed} as possible
\vfill
this problem relates to, and can be used to solve, other problems:
\textit{minimum fill-in, maximum clique, graph coloring, maximum independent sets}
\end{frame}
  

\begin{frame}{ }
\Huge \smiley
\end{frame}

\end{document}


